%----------------------------------------------------------------------------
\chapter{\bevezetes}
%----------------------------------------------------------------------------

A világ az egyre növekvő automatizáltság felé halad, de a múltban használt klasszikus programozási módszerek nem alkalmasak minden feladat megoldására. Itt jön képbe a gépi tanulás, mely ezen feladatokban áttöréseket ért el a közelmúltban, például az autonóm autó technológiák terén. Viszont a gépi tanulási módszereknek nem rendelkeznek a klasszikus programozás egy fontos jellemzőjével, ami a magyarázhatóság. Egy konkrét célra készített programnál a fejlesztők pontosan tudják, hogy milyen módon áll elő a kimenet, így sok bizonyítható, hogy az megfelelő minőséggel oldja-e meg feladatát (implementációs hibáktól eltekintve). Gépi tanulásnál viszont általános algoritmusokról beszélünk, melyeket egy konkrét feladatra megtanítva ez az információ a belső paraméterekbe van belekódolva, emberileg értelmezhetetlen módon. Ez alól vannak kivételek, például a lineáris modellek, vagy a döntési fák, viszont összetett problémáknál a gyakorlatban bonyolultabb megoldások az elterjedtek. 
	
Ha a gépi tanuló módszerek értelmezhetőek lennének, az több szempontból is hasznos lenne. Rávilágíthatnak olyan hibákra, amiket a példákon való tesztelés nem hoz felszínre, így javítva a rendszerek megbízhatóságát. Bizalmat kelthet a rendszerek felhasználóiban, azáltal hogy jobban megértik a működését. Vagy meggyorsíthatja a fejlesztési folyamatokat, azáltal, hogy korán rávilágít problémákra.

A magyarázatgenerálás célja az értelmezhetőség megteremtése komplexebb gépi tanulási modelleknél is. Ezen cél elérésére több különböző módszer létezik. Vannak globális módszerek, melyek például megadják a bemeneti paraméterek fontosságát a hálóban. Ezek hiányossága viszont, hogy általános képet festenek, ami nem veszi figyelembe az attribútumok interakcióját. A feladat megközelíthető lokális módon is, azaz a magyarázat egy bemenet kiértékelésére vonatkozik. Ezzel egyszerre csak egy kis részét vizsgálhatjuk a modellnek, de több példa felhasználásával közelebb juthatunk egy átfogó magyarázathoz. A dolgozat fókuszában ezen lokális megközelítés egy megvalósítása áll, a LIME (Linear Model-Agnostic Explanations) rendszer, ami lokális model-agnosztikus, azaz bármely gépi tanulási módszerrel kompatibilis magyarázatokat állít elő a modell viselkedésének lokális közelítésével.

\section{Szakdolgozat célja}

A dolgozat fő célja ismertetni működési alapelvét és használatát a LIME rendszernek, illetve megvizsgálni működését egy neurális hálózaton.

Az önálló laboratórium keretében végzett munkám során a depresszió genetikai komponensét vizsgáltam neurális háló segítségével. Az adathalmaz zajossága és mérete miatt ez nehézségekbe ütközött. Később az adathalmazt kiegészítették egy ismert modell alapján mesterségesen generált célváltozóval, amely egyrészt kevésbé zajos, másrészt fel lehet használni kiértékelésre. Mivel ez a célváltozó hasonló jellegű mint az eredeti valós megfelelője, segíthet olyan paraméterek és módszerek megtalálásában, melyekkel az eredeti célváltozó is jobb eredménnyel approximálható.

Ezáltal a dolgozat másodlagos célja optimális pontosságú hálózat kialakítása ezen mesterséges célváltozóra. A két feladat azon módon kapcsolódik, hogy a LIME rendszert ezen hálózattal fogom tesztelni. Ez azért is előnyös kombináció, mivel rendelkezésre állnak azon szabályok, melyek alapján a célváltozó generálva lett, így ellenőrizhető a LIME által visszaadott szignifikáns attribútumok validitása. Ezen kívül az attribútumok nagy száma (\~2400) segít megvizsgálni, hogy szélsőségesen sok attribútum esetén hogy viselkedik a LIME rendszer.

\section{Szakdolgozat felépítése}

A dolgozat először az elméleti alapokat tárgyalja. \chapref{ch:nn} ismerteti a neurális hálók általános működését és azon általános építőelemeit, melyek a dolgozatban bemutatott háló részei. Ezután \chapref{ch:exp} a magyarázatgenerálást taglalja általánosságban. 

\chapref{ch:lime} a LIME rendszer elméleti hátterét és felépítését tárgyalja.

\chapref{ch:example} a lefektetett elméleti alapok és rendszerek gyakorlati működését írja le. Bemutatja a lime rendszer működését egy külső példa adathalmaz segítségével, majd a fentebb ismertetett saját adathalmazon. Továbbá a fejezet ismerteti a tanítási folyamatát is az ezen adathalmazra alkalmazott neurális hálónak.



