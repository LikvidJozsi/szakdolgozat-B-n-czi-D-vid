\pagenumbering{roman}
\setcounter{page}{1}

\selecthungarian

%----------------------------------------------------------------------------
% Abstract in Hungarian
%----------------------------------------------------------------------------
\chapter*{Kivonat}\addcontentsline{toc}{chapter}{Kivonat}

A gépi tanuló rendszerek gyakorlati alkalmazásának egyik nehézsége az értelmezhetőségük. A mai komplex modellek működése fekete doboz jellegű, azaz a bemenet és kimenet könnyen megérthető, de a köztük lévő leképezés nem értelmezhető. Más szóval, látjuk hogy egy modell milyen kimenetet ad, de azt nem hogy miképp döntött így. Ha nem áll rendelkezésre megfelelő méretű és elég változatos teszt adathalmaz, akkor előfordulhat, hogy megfelelőnek ítélünk hibás modelleket. Ezen problémát hivatott megoldani a magyarázat generálás, mely megpróbálja meghatározni, hogy a modell a bemenet mely tulajdonságaira alapozta döntését.
	 
A LIME (Linear Model-Agnostic Explanations) magyarázat generáló rendszer egy általános, tetszőleges gépi tanulási megoldás magyarázatára alkalmas keretrendszer. Ezen dokumentum ismerteti működési alapelvét és használatát, illetve megvizsgálja teljesítményét egy neurális hálózaton. Ezen neurális háló nagy számú tulajdonságot tartalmazó genetikai adathalmaz mesterségesen generált célváltozójára lett tanítva. A tulajdonságok nagy száma egyedi kihívást jelent a háló tanítása szempontjából, és szemlélteti a LIME rendszer működését szélsőséges paraméterek esetében. 


\vfill
\selectenglish


%----------------------------------------------------------------------------
% Abstract in English
%----------------------------------------------------------------------------
\chapter*{Abstract}\addcontentsline{toc}{chapter}{Abstract}

TODO


\vfill
\selectthesislanguage

\newcounter{romanPage}
\setcounter{romanPage}{\value{page}}
\stepcounter{romanPage}